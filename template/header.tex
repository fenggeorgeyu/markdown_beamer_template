\usepackage[utf8]{inputenc}
\usepackage{hyperref}
\usepackage{animate}
\usepackage{multicol}%multi columns e.g. \begin{multicols}[2] ... \end{multicols}
\usepackage{multirow}
\usepackage{booktabs} % table \toprule \middlerule \bottomrule
\usepackage[pdf]{graphviz} % graphviz in latex
% --- subfigure --- 
\usepackage{subfigure, subfloat}
%\usepackage[export]{adjustbox} % vlignment for vertical alignment for \subfigure commands

%%---tikz---
\usepackage{tikz}
\usetikzlibrary{shapes,arrows,chains,positioning}



%%---animation on tikz tree nodes---
%%https://tex.stackexchange.com/questions/6135/how-to-make-beamer-overlays-with-tikz-node
\tikzset{onslide/.code args={<#1>#2}{%
  \only<#1>{\pgfkeysalso{#2}} % \pgfkeysalso doesn't change the path
}}
\tikzset{temporal/.code args={<#1>#2#3#4}{%
  \temporal<#1>{\pgfkeysalso{#2}}{\pgfkeysalso{#3}}{\pgfkeysalso{#4}} % \pgfkeysalso doesn't change the path
}}

%%---flow char examples---
%% https://tex.stackexchange.com/questions/274280/building-a-flow-chart-in-beamer-using-tikz
%%---tikz flow chart examples---
%% http://www.texample.net/tikz/examples/tag/flowcharts/


% --- coding---
\usepackage{fancybox}
\usepackage{xcolor}
% \usepackage{times}
% \usepackage{listings}

% allow frame break by default
% \let\oldframe\frame
% \renewcommand\frame[1][allowframebreaks]{\oldframe[#1]} 

%==================new command=================
\newcommand{\nt}[1]{\begin{theorem}#1\end{theorem}}
\newcommand{\npr}[1]{\begin{proposition}#1\end{proposition}}
\newcommand{\nc}[1]{\begin{corollary}#1\end{corollary}}
\newcommand{\nlem}[1]{\begin{lemma}#1\end{lemma}}
\newcommand{\nd}[1]{\begin{definition}#1\end{definition}}%\begin{flushright}
%\IEEEQED\end{flushright}
\newcommand{\nexp}[1]{\begin{example}#1\end{example}}
\newcommand{\npf}[1]{\begin{proof}#1\end{proof}}
\newcommand{\nequ}[1]{\begin{equation}#1\end{equation}}
\newcommand{\nrmk}[1]{\begin{remark}#1\end{remark}}


%%--other
\newcommand{\enum}[1]{\begin{enumerate}#1\end{enumerate}}   % list items
\newcommand{\al}[1]{\begin{align*}#1\end{align*}} % align equations no numbering
\newcommand{\red}[1]{{\color{red}#1}}

\newcommand{\eps}{\epsilon} % epsilon symbol for permutation
%---vector, more flexible---
%\newcommand{\vect}[1]{\boldsymbol{#1}} % bold style
%\newcommand{\vect}[1]{\overset{\rightharpoonup}#1} % half arrow style
\newcommand{\vect}[1]{\vv{#1}}


\newcommand{\vectint}[2]{[\vect{#1},\vect{#2}]} % vector interval e.g. [\vect{s},\vect{e}]

\usepackage{mathtools}
\DeclarePairedDelimiter{\ceil}{\lceil}{\rceil} % ceiling function
\DeclarePairedDelimiter{\floor}{\lfloor}{\rfloor} % floor function

%---stanford & dartmouth relational operator

\def\ojoin{\setbox0=\hbox{$\bowtie$}%
  \rule[-.02ex]{.25em}{.4pt}\llap{\rule[\ht0]{.25em}{.4pt}}}
\def\leftouterjoin{\mathbin{\ojoin\mkern-5.8mu\bowtie}}
\def\rightouterjoin{\mathbin{\bowtie\mkern-5.8mu\ojoin}}
\def\fullouterjoin{\mathbin{\ojoin\mkern-5.8mu\bowtie\mkern-5.8mu\ojoin}}
\newcommand{\extjoin}{\fullouterjoin_+}


% Relational algebra symbols from ftp://reports.stanford.edu/www/dbgroup_only/latex-macros.html
\def\select{\mbox{\Large$\sigma$}}
\def\cross{\mbox{$\times$}}
\def\intersection{\mbox{$\cap$}}
\def\intersect{\mbox{$\cap$}}
\def\union{\mbox{$\cup$}}
\def\join{\mbox{$\bowtie$}}
\def\leftsemijoin{\mbox{$\mathrel{\raise1pt\hbox{\vrule height6pt depth0pt width0.6pt\hskip-1.5pt$>$\hskip -2.5pt$<$}}$}}
\def\rightsemijoin{\mbox{$\mathrel{\raise1pt\hbox{\hskip-1.5pt$>$\hskip -2.5pt$<$\hskip -1.5pt\vrule height6pt depth0pt width0.6pt}}$}}
\def\project{\mbox{\Large$\pi$}}
\def\Project{\mbox{$\Pi$}}
\def\groupby{\mbox{\Large$G$}}

%---outer join---
\def\ojoin{\setbox0=\hbox{$\bowtie$}\rule[-.02ex]{.25em}{.4pt}\llap{\rule[\ht0]{.25em}{.4pt}}}
\def\leftouterjoin{\mathbin{\ojoin\mkern-5.8mu\bowtie}}
\def\rightouterjoin{\mathbin{\bowtie\mkern-5.8mu\ojoin}}
\def\outerjoin{\mathbin{\ojoin\mkern-5.8mu\bowtie\mkern-5.8mu\ojoin}}
\def\extouterjoin{\outerjoin_+} % extended outer join

%---map reduce---
\def\mapjoin{\overset{\text{\tiny{map}}}{\join}}
\def\leftmapjoin{\overset{\text{\tiny{map}}}{\leftouterjoin}}
\def\rightmapjoin{\overset{\text{\tiny{map}}}{\rightouterjoin}}
\def\reducejoin{\overset{\text{\tiny{reduce}}}{\join}}